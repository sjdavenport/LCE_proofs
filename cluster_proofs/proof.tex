\documentclass[11pt]{article}

\usepackage[utf8]{inputenc}
\usepackage[T1]{fontenc}
\usepackage{amsmath,amsthm,amssymb}
\usepackage[margin=1in]{geometry}
\usepackage{enumitem}

% Theorem environments
\newtheorem{theorem}{Theorem}
\newtheorem{corollary}[theorem]{Corollary}
\newtheorem{lemma}[theorem]{Lemma}
\newtheorem{definition}[theorem]{Definition}

\theoremstyle{remark}
\newtheorem{remark}{Remark}

\title{Superthreshold Clusters Containing TFCE-Significant Clusters are LCE-Significant}
\author{}
\date{}

\begin{document}

\maketitle

\section{Preliminaries}

We work within the framework established in the main paper. Let $\mathcal{B} \subset \mathbb{R}^3$ denote the set of voxels making up the brain, and for each voxel $v \in \mathcal{B}$, let $T_v$ denote the test statistic. The excursion set at threshold $h$ is $\mathcal{E}(h) = \{v \in \mathcal{B} : T_v \geq h\}$.

For the generalized TFCE statistic, we have:
\[
S_v = \int_{h_0}^{\infty} f(h) g(e_v(h)) \, dh
\]
where $e_v(h) = |C_v(h)|$ is the size of the connected component of $\mathcal{E}(h)$ containing $v$ (or zero if $v \notin \mathcal{E}(h)$). We assume $g$ is non-decreasing with $g(0) = 0$, and $f(h) \geq 0$ for $h \geq h_0$.

\begin{definition}
An \textbf{$h_0$-superthreshold cluster} is a maximal connected component of the set $\{v \in \mathcal{B} : T_v \geq h_0\}$.
\end{definition}

\begin{definition}
A voxel $v$ is \textbf{TFCE-significant} if $S_v(\mathbf{X}_\mathcal{B}) \geq t^*_\mathcal{B}$, where $t^*_\mathcal{B}$ is the permutation-based critical value. A \textbf{TFCE-significant cluster} is a connected set of TFCE-significant voxels.
\end{definition}

\begin{definition}
A region $R \subseteq \mathcal{B}$ is \textbf{LCE-significant} if $\max_{v \in R} S_v(\mathbf{X}_R) > t^*_\mathcal{B}$, where $\mathbf{X}_R$ denotes the masked data with voxels outside $R$ set to zero.
\end{definition}

\section{Main Result}

\begin{theorem}\label{thm:main}
Let $R$ be a TFCE-significant cluster. Then any $h_0$-superthreshold cluster $R' \supseteq R$ is LCE-significant.
\end{theorem}

\begin{proof}
Let $R$ be a TFCE-significant cluster. By definition, for all $v \in R$ we have $S_v(\mathbf{X}_\mathcal{B}) \geq t^*_\mathcal{B}$.

\medskip
\noindent\textbf{Step 1: TFCE-significant voxels lie in $h_0$-superthreshold clusters.}

\medskip
By the assumptions on $f$ and $g$, we have $S_v \geq 0$ for all voxels and all permutations, so $t^*_\mathcal{B} > 0$. 

For any voxel $v$ with $T_v < h_0$, we have $e_v(h) = 0$ for all $h \geq h_0$ since $v \notin \mathcal{E}(h)$ when $T_v < h$. Therefore:
\[
S_v(\mathbf{X}_\mathcal{B}) = \int_{h_0}^{\infty} f(h) g(e_v(h)) \, dh = \int_{h_0}^{\infty} f(h) g(0) \, dh = 0
\]
since $g(0) = 0$. This means $S_v(\mathbf{X}_\mathcal{B}) = 0 < t^*_\mathcal{B}$, so $v$ cannot be TFCE-significant.

By contraposition, every TFCE-significant voxel $v \in R$ must satisfy $T_v \geq h_0$. Since $R$ is connected, there exists an $h_0$-superthreshold cluster $R' \supseteq R$.

\medskip
\noindent\textbf{Step 2: TFCE statistics are unchanged after masking to $R'$.}

\medskip
Let $R'$ be any $h_0$-superthreshold cluster containing $R$. For any $v \in R'$ and any $h \geq h_0$, we claim that:
\[
e_v(h, \mathbf{X}_{R'}) = e_v(h, \mathbf{X}_\mathcal{B}).
\]

To establish this, observe that $C_v(h)$, the connected component of the excursion set $\mathcal{E}(h) = \{w \in \mathcal{B} : T_w \geq h\}$ containing $v$, must satisfy $C_v(h) \subseteq R'$. This follows because:
\begin{itemize}[leftmargin=2em]
    \item All voxels in $\mathcal{B} \setminus R'$ that are neighbors of $R'$ have $T_w < h_0 \leq h$ (by the definition of $R'$ being an $h_0$-superthreshold cluster).
    \item Therefore, the connected component $C_v(h)$ cannot extend beyond $R'$.
\end{itemize}

Since masking to $R'$ does not affect the cluster $C_v(h)$, we have:
\[
e_v(h, \mathbf{X}_{R'}) = |C_v(h)| = e_v(h, \mathbf{X}_\mathcal{B}).
\]

Consequently, for all $v \in R'$:
\[
S_v(\mathbf{X}_{R'}) = \int_{h_0}^{\infty} f(h) g(e_v(h, \mathbf{X}_{R'})) \, dh = \int_{h_0}^{\infty} f(h) g(e_v(h, \mathbf{X}_\mathcal{B})) \, dh = S_v(\mathbf{X}_\mathcal{B}).
\]

\medskip
\noindent\textbf{Step 3: Concluding LCE significance.}

\medskip
Since $R \subseteq R'$, there exists $v \in R \subseteq R'$ such that:
\[
S_v(\mathbf{X}_{R'}) = S_v(\mathbf{X}_\mathcal{B}) \geq t^*_\mathcal{B}.
\]

Therefore:
\[
\max_{v \in R'} S_v(\mathbf{X}_{R'}) \geq t^*_\mathcal{B}.
\]

By the definition of LCE significance, $R'$ is LCE-significant.
\end{proof}

\begin{remark}
The key insight of this proof is that $h_0$-superthreshold clusters act as ``islands'' in the test statistic map: when masking to such a cluster, the TFCE statistics for voxels inside the cluster remain unchanged because the cluster boundaries (where $T_v < h_0$) prevent any information from voxels outside the region from influencing the statistic. This is precisely why the threshold $h_0$ plays such a crucial role in determining the spatial support of valid TFCE inference.
\end{remark}

\begin{corollary}
For every TFCE-significant cluster $R$, there exists a unique minimal $h_0$-superthreshold cluster $R' \supseteq R$ that is LCE-significant. This cluster $R'$ is precisely the connected component of $\{v \in \mathcal{B} : T_v \geq h_0\}$ containing $R$.
\end{corollary}

\end{document}
